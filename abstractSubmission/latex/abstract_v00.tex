\documentclass[12pt]{article}

\usepackage[a4paper, total={6in, 10in}]{geometry}

\author{Miguel P Xochicale, Chris Baber \\
% \author{Miguel P Xochicale \qquad Chris Baber \\
\{map479, c.baber\}@bham.ac.uk \\
Department of Electronic, Electric and System Engineering\\
School of Engineering\\
University of Birmingham, UK}
% \title{Towards the Automatic Measurement of Variability in Human Movement and Emotions} %(Friday, 1th December 2017)
% Variability in Using Nonlinear Time Series Analysis %(Saturday, 2th December 2017)
% \title{Nonlinear Dynamics in Human Movement and Face Emotions} %(Saturday, 2th December 2017)
\title{Towards the Analysis of Nonlinear Dynamics in Human Movement and Face Emotions} %(Sunday, 3th December 2017)


\date{\today}

\begin{document}
\maketitle

%%%%%%%%%%%%%%%%%%%%%%%%%%%%%%%%%%%%%%%%%%%%%%%%%%%%%%%%%%%%%%%%%%%%%%%%%%%%%%%%
% Abstract of the proposed open paper: a concise and clear description of its
% objectives, methods, results, and conclusions (with a maximum of 300 words).



% (stergius marylin univeristy movemetn varilbty)
% Stergiou from Nebraska University
% Stergiou et al. 2011 \cite{stergiou2011}
% states that movement variability is an inherent feature between and within persons.
% Such movement variability has been quantified in the previous three decades using
% nonlinear dynamics.

% In the previous three decades a body of work has been well established
% about the unerstanding and measuramente of movement varialibyt
% in which variability is an inherent feature between and within persons.

The understanding and measurement of movement variability has been well established
in the previous three decades in areas such as biomechanics, sport science
and lately in human-robot interaction. These interested is raised because
variability is an inherent feature between and within persons.
Such premise lead us to hypothesise that not only movement variability
can be quantified but also the subtle variations of face emotions across.
In our previous studies of Human-Humanoid Imitation Activities using nonlinear
time series analysis, we explored the quantification of variability of movement
of upper arms and head pose estimation within and between participants.
In those experiments, we observed that not only the variability of upper arms
movements were an important feature in the human-humanoid imitation activities
but also the variability of face emotion across time.
For this work, we present preliminary results of the use of the state reconstruction
to understand the relationship of variability between arm movements, head pose
estimation and face emotion.
Particularly, we considered the variability of time series in the face landmarks
of 18 participants to create a multidimensional representation in the state space
reconstruction. These preliminary results of the state space reconstruction lead
us to conclude that not only arm movement variability
can be analysed and quantified
but also the subtle variabilitu of face emotions can be represented as a nonlinear
dynamic system where the emotions are not a discrete system that goes from emotion transitions
(e.g. excitement to neutral to boredom, etc)
but as dynamic process that subtly changes across time.
% tiny-sensible-

 % and be better represented.
% require a better understanding

% For the results, we show that not only the variablity of movement can be quantified
% but also the variation of emotions across time.
%
% we proposed the use of a tool in nonlinear dynamics, the state space reconstruction,
% to measure the variability of simple arm movement
% with that we realise that non only the arm movement is affected in the interaction
% but also the variability of emotions presented in the facial expressions of the
% participants. In this open paper, we will show results of an experiment where
% 20 participants were asked to imitate NAO, a humanoid robot, for 20 senconds in
% order to evaluate how the movement of arms and head pose estimatation variated
% across time.





%% EXTRA IDEA: human movement is not noise
% Recently, Herzfeld et al. \cite{Herzfeld2014}
% conducted experiments to state that movement variability is not only noise but a
% source of movement exploration which at certain point of the exploration
% such variability is becoming a source of movement exploration.




% 2. Equations can be included using LaTeX syntax:
% ```
% \begin{equation}\label{eq1}
% E=mc^2
% \end{equation}
% ```
% and refer to the equation using Eq.~\ref{eq1}.

% 3. Special symbols in the abstract should be edited in LaTeX syntax.
%
% 4. References can be added into the abstract:
%    Use the LaTeX citation command \cite{key} and add at
%    the end of the abstract:
%
%



% \begin{thebibliography}{10}
% % \bibitem{key}
% % F. Alonso, R. Nadal, {\it Phys. Rev. Lett.} {\bf 108} 123456 (2012)
%
% \bibitem{stergiou2011}
% Stergiou N, Decker LM., {\it Human Movement Variability, Nonlinear Dynamics,
%  and Pathology: Is There A Connection?}, Human Movement Science 30(5), 869–888, (2011)
%
% \end{thebibliography}


\end{document}
