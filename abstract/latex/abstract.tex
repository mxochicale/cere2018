\documentclass[12pt]{article}

\usepackage[a4paper, total={6in, 10in}]{geometry}

\author{Miguel P Xochicale, Chris Baber \\
% \author{Miguel P Xochicale \qquad Chris Baber \\
\{map479, c.baber\}@bham.ac.uk \\
Department of Electronic, Electric and System Engineering\\
School of Engineering\\
University of Birmingham, UK}
% \title{Towards the Automatic Measurement of Variability in Human Movement and Emotions} %(Friday, 1th December 2017)
% Variability in Using Nonlinear Time Series Analysis %(Saturday, 2th December 2017)
% \title{Nonlinear Dynamics in Human Movement and Face Emotions} %(Saturday, 2th December 2017)
\title{Towards the Analysis of Nonlinear Dynamics in Human Movement and Face Emotions} %(Sunday, 3th December 2017)


\date{\today}

\begin{document}
\maketitle

%%%%%%%%%%%%%%%%%%%%%%%%%%%%%%%%%%%%%%%%%%%%%%%%%%%%%%%%%%%%%%%%%%%%%%%%%%%%%%%%
% Abstract of the proposed open paper: a concise and clear description of its
% objectives, methods, results, and conclusions (with a maximum of 300 words).

% Stergiou from Nebraska University
% Stergiou et al. 2011 \cite{stergiou2011}


Movement variability is an inherent feature within and between persons
and understanding and measurement of movement variability has been well established
in the previous three decades mainly in areas such as biomechanics, sport science
and much recently in human-robot interaction.
With that in mind, we hypothesise that the subtle variations of face emotions
can be described and quantified in a similar fashion as with the methodologies
of movement variability.
Such methodologies are based on nonlinear dynamics, particularly the state space
reconstruction theorem where the dynamics of an unknown system can be reconstructed
using one dimensional time series.
For this work, we explain how the state reconstruction theorem works and
present preliminary results of the use of the state reconstruction
to understand the relationship of variability between arm movements, head pose
estimation and face emotion.
Particularly, we focus our work on the variability of time series in the face landmarks
of 18 participants to create a multidimensional representation in the state space
reconstruction. These preliminary results of the state space reconstruction
in the context of face emotions lead us to conclude that not only
the variability of upper body movement can be analysed and quantified
but also the subtle variability of face emotion transitions
across time (e.g. excitement to neutral to boredom, etc)
can be understood and measured using nonlinear dynamics.




\end{document}
