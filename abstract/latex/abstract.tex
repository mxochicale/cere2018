\documentclass[12pt]{article}
\usepackage[a4paper, total={6in, 10in}]{geometry}
\usepackage{pgfgantt}

\author{Miguel P Xochicale, Chris Baber \\
% \author{Miguel P Xochicale \qquad Chris Baber \\
\{map479, c.baber\}@bham.ac.uk \\
Department of Electronic, Electric and System Engineering\\
School of Engineering\\
University of Birmingham, UK}
\title{Towards the Analysis of Movement Variability For Facial Expressions with
Nonlinear Dynamics} 
\date{\today}

\begin{document}
\maketitle

\section*{Abstract}
Movement variability is an inherent feature within and between persons.
Research on measurement and understanding of movement variability has been well
established in the previous three decades in areas such as biomechanics,
sport science, psychology, cognitive science, neuroscience and robotics.
With that in mind, we hypothesise that the subtle variations of facial expressions
can be described and quantified in a similar fashion as with the methodologies
of movement variability.
Such methodologies are based on nonlinear dynamics, particularly with the use of
the state space reconstruction theorem where dynamics of an unknown system
can be reconstructed using one dimensional time series.
For this work, we explain how the state space reconstruction theorem works and
present preliminary results of the use of the state reconstruction
to understand the relationship between the variability of arm movements, head
pose estimation and facial expressions of six participants.
The results of the state space reconstruction in the context of facial expressions
lead us to conclude that not only the variability of upper body movement can be
analysed and quantified but also the subtle variability of facial expressions
across time (e.g. variation of facial expressions from excitement to neutral to boredom, etc)
can be understood and measured using nonlinear dynamics.


\section*{Gantt Chart}

\begin{ganttchart}[
	hgrid,
	vgrid,
	x unit=2mm,
	time slot format=isodate-yearmonth
	]{2018-03-01}{2018-04-30}
\gantttitlecalendar{year,month=shortname} \\
%\ganttmilestone{training at BrH}{2018-03-26} \\
\ganttbar{Slides Preparation}{2018-03-01}{2018-04-03} \\
\ganttbar{Conference}{2018-04-04}{2018-04-05} 
%\ganttmilestone{conf}{2018-04-04}
 
\end{ganttchart}




\end{document}
