\documentclass[compress]{beamer}

%--------------------------------------------------------------------------
% Common packages
%--------------------------------------------------------------------------
\usepackage[english]{babel}
\usepackage{pgfpages} % required for notes on second screen
\usepackage{graphicx}
\usepackage{subfigure}
\usepackage{multicol}
\usepackage[normalem]{ulem}

\usepackage{tabularx,ragged2e}
\usepackage{booktabs}
\usepackage{marvosym}


\usepackage{fontawesome}
% \usepackage[tt=false, type1=true]{libertine}
% \usepackage[T1]{fontenc}



\usepackage{tikz}
\usetikzlibrary{calc,shapes,shadows}
\usetikzlibrary{fadings}



%--------------------------------------------------------------------------
% Load theme
%--------------------------------------------------------------------------
\usetheme{hri}


\graphicspath{{figs/}}


%--------------------------------------------------------------------------
% General presentation settings
%--------------------------------------------------------------------------
\title{Towards the Analysis of Movement Variability for Facial Expressions  \\ 
	with Nonlinear Dynamics} 
\subtitle{CERE 2018}
\date{Glasgow, Scotland, 4-5 April 2018}

\author{Miguel P Xochicale and Chris Baber}
\institute{School of Engineering \\{\bf University of Birmingham}}




%--------------------------------------------------------------------------
% Notes settings
%--------------------------------------------------------------------------
%\setbeameroption{show notes on second screen}
%\setbeameroption{hide notes}






\begin{document}


%%%%%%%%%%%%%%%%%%%%%%%%%%%%%%%%%%%%%%%%%%%%%%%%%%%%%%%%
\licenseframe{https://github.com/mxochicale/cere2018}

%%%%%%%%%%%%%%%%%%%%%%%%%%%%%%%%%%%%%%%%%%%%%%%%%%%%%%%%
\maketitle





%--------------------------------------------------------------------------
% Content
%--------------------------------------------------------------------------
%
\section{Movement Variability}


\subsection{}
%%%%%%%%%%%%%%%%%%%%%%%%%%%%%%%%%%%%%%%%%%%%%%%%%%%%%%%%
{

%\paper{Newell K M, Corcos D M, {\bf Variability and Motor Control}, 1993}
\paper{Lockhart T, Stergiou N, {\bf New Perspectives in Human Movement Variability}, 2013}

\begin{frame}{What is Movement Variabily?}
MOVEMENT VARIABILITY is defined as the variations that occur in motor
performance across multiple repetitions of a task and such behaviour is 
an inherent feature within and between each person's movement.
\end{frame}
}



\section{State Space Reconstruction}

%\subsection{}
%%%%%%%%%%%%%%%%%%%%%%%%%%%%%%%%%%%%%%%%%%%%%%%%%%%%%%%
\imageframe[caption=Uniform Time-Delay Embedding]{utde/utde}


%\subsection{}
%%%%%%%%%%%%%%%%%%%%%%%%%%%%%%%%%%%%%%%%%%%%%%%%%%%%%%%
\imageframe[caption=Minimum Embedding Dimension ($m$) ]{utde/cao}


%\subsection{}
%%%%%%%%%%%%%%%%%%%%%%%%%%%%%%%%%%%%%%%%%%%%%%%%%%%%%%%
\imageframe[caption=Minimum Embedding Delay ($\tau$)]{utde/ami}





\section{Experiment and Results}

\subsection{}
\imageframe[caption=Human-Robot Imitation Activity]{hri/experiment}

\subsection{Arm Movement Variability}
%%%%%%%%%%%%%%%%%%%%%%%%%%%%%%%%%%%%%%%%%%%%%%%%%%%%%%%
\imageframe[caption=Participants]{hri/participants}


\subsection{IMU Time Series}
%%%%%%%%%%%%%%%%%%%%%%%%%%%%%%%%%%%%%%%%%%%%%%%%%%%%%%%
\imageframe[caption=Time Series for IMU]{timeseries/razor}



%%%%%%%%%%%%%%%%%%%%%%%%%%%%%%%%%%%%%%%%%%%%%%%%%%%%
\subsection{Reconstructed State Spaces}
\imageframe[caption=RSS ($m 10$ $\tau 10$) for sgzmuvAx]{results/ssrrazor}


%%%%%%%%%%%%%%%%%%%%%%%%%%%%%%%%%%%%%%%%%%%%%%%%%%%%
\subsection{Reconstructed State Spaces}
\imageframe[caption=Euclidean Distances ($m 10$ $\tau 10$) for IMU ]{results/edrazor}


\subsection{OpenFace Time Series}
%%%%%%%%%%%%%%%%%%%%%%%%%%%%%%%%%%%%%%%%%%%%%%%%%%%%%%
\imageframe[caption=Time Series for OpenFace pose Tx]{timeseries/openface_Tx}


%%%%%%%%%%%%%%%%%%%%%%%%%%%%%%%%%%%%%%%%%%%%%%%%%%%%%
\subsection{OpenFace Time Series}
\imageframe[caption=Time Series for OpenFace Landmarks]{timeseries/openface_x067}


%%%%%%%%%%%%%%%%%%%%%%%%%%%%%%%%%%%%%%%%%%%%%%%%%%%%
\subsection{Reconstructed State Spaces}
\imageframe[caption=RSS ($m 10$ $\tau 10$) for zmuvx\_0]{results/ssropenface}


%%%%%%%%%%%%%%%%%%%%%%%%%%%%%%%%%%%%%%%%%%%%%%%%%%%%
\subsection{Reconstructed State Spaces}
\imageframe[caption=Euclidean Distances ($m 10$ $\tau 10$) for Openface]{results/edopenface}





%
%\subsection{Time Varying Facial Expressions}
%%%%%%%%%%%%%%%%%%%%%%%%%%%%%%%%%%%%%%%%%%%%%%%%%%%%%%%%%
%
%
%{
%\begin{frame}{Time Varying Facial Expressions}
%
%
%\end{frame}
%}
%


%
%\subsection{}
%%%%%%%%%%%%%%%%%%%%%%%%%%%%%%%%%%%%%%%%%%%%%%%%%%%%%%%%%
%{
%\begin{frame}{TODO List}
%
%
%          \begin{enumerate}
%              \item 1
%              \item 2 
%              \item 3 
%          \end{enumerate}
%
%%\badge{/badge/logo_badge}
%\end{frame}
%}
%
%




\section{Conclusions and Future Work}

\subsection{}
%%%%%%%%%%%%%%%%%%%%%%%%%%%%%%%%%%%%%%%%%%%%%%%%%%%%%%%%
{
\begin{frame}{Conclusions Future Work}

\begin{itemize}
	\item Quantification Arm Movement Variability with Nonlinear Dynamics
        \item However,
        \item Timeseries from the landmarks are mounted on the pose location of the head. 
\end{itemize}

\begin{itemize}
	\item Test other techniques of Nonlinear Dynamics, e.g. Lyapunov Exponents, Recurrent Quantification Analysis
	\item Use of Convolutional Neural Networks for automatic identification of Movement Variability
\end{itemize}



%\badge{/badge/logo_badge}
\end{frame}
}






%%%%%%%%%%%%%%%%%%%%%%%%%%%%%%%%%%%%%%%%%%%%%%%%%%%%%%%%
\begin{frame}{Bibliography}
    \begin{thebibliography}{10}

\beamertemplatearticlebibitems
  \bibitem{Ho2016}
      Jostine Ho       
      \newblock \doublequoted{Facial Emotion Recognition}
      \newblock GitHub repository (2016), https:// github.com/JostineHo/mememoji [\href{https:// github.com/JostineHo/mememoji}{\faGithub}]


    \end{thebibliography}
\end{frame}




%%%%%%%%%%%%%%%%%%%%%%%%%%%%%%%%%%%%%%%%%%%%%%%%%%%%%%%%
\closingtitle


%%%%%%%%%%%%%%%%%%%%%%%%%%%%%%%%%%%%%%%%%%%%%%%%%%%%%%%%
\licenseframe{https://github.com/mxochicale/cere2018}



\end{document}
